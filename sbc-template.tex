\documentclass[12pt]{article}

\usepackage{sbc-template}
\usepackage{graphicx,url}
\usepackage[utf8]{inputenc}
\usepackage[brazil]{babel}
\usepackage{verbatim}
%\usepackage[latin1]{inputenc}

% 19/06: http://sbrt.org.br/sbrt2020/cfp.html#cfp
% 01/07: http://voyager.ce.fit.ac.jp/conf/3pgcic/2020/
% 10/07: http://sbseg.sbc.org.br/2020/pt/chamadas/principal.html
% 14/07: http://conferences.cis.umac.mo/pdcat20/
     
\usepackage{xspace}
\newcommand{\FC} {Freechains\xspace}

\newcommand{\Xon} {$1{\rightarrow}N$\xspace}
\newcommand{\Xno} {$1{\leftarrow}N$\xspace}
\newcommand{\Xnn} {$N{\leftrightarrow}N$\xspace}
\newcommand{\Xoo} {$1{\leftrightarrow}1$\xspace}
\newcommand{\Xo}  {$1{\hookleftarrow}$\xspace}

\sloppy

\begin{comment}
http://sbseg.sbc.org.br/2020/pt/chamadas/ferramentas.html

A survey of peer-to-peer content distribution technologies.

- Anonimização e privacidade
- Controle de acesso, autenticação, biometria, confiança e gestão de identidades
- Criptografia e criptoanálise: algoritmos, protocolos e aplicações
- Criptomoedas e mecanismos de consenso distribuído

- Descrição e motivação do problema resolvido pela ferramenta;
- Arquitetura da solução e descrição das principais funcionalidades;
- URL onde a ferramenta está disponível (o código-fonte com a licença deve estar presente nessa URL, não apenas o binário);
- URL da documentação da ferramenta, incluindo informações e requisitos para instalação;
- Descrição da demonstração planejada para o Salão de Ferramentas, informando equipamentos necessários para tal;
- (opcional) URL com um vídeo explicando a instalação e as funcionalidades da ferramenta.

- Até 8 páginas

    - public key cryptography to deal with identities, confidentiality and
\end{comment}


\title{\FC: Disseminação de Conteúdo Peer-to-Peer}

\author{Francisco Sant'Anna\inst{1}}
\address{UERJ - Departamento de Ciência da Computação}

\begin{document} 

\maketitle

\begin{abstract}
O \FC é um sistema peer-to-peer para disseminação de conteúdo: um
usuário posta uma mensagem em um tópico e seus assinantes eventualmente recebem
a mensagem.
As postagens são estruturadas em um grafo direcionado acíclico criptográfico
que é imune a modificações (\emph{Merkle DAG}).
O grafo é disseminado entre os pares da rede por \emph{gossip} e de maneira não
estruturada.
O \FC suporta múltiplos arranjos de disseminação pública e privada entre grupos
e indivíduos, sendo possível modelar desde conversas privadas por e-mail, até
debates entre desconhecidos em fóruns públicos.
Cada tópico público conta com um sistema descentralizado de reputação para
combater abusos, tais como SPAM e notícias falsas.
%Autores precisam de reputação prévia para postar novo conteúdo, que ainda assim
%pode ser removido se a proporção entre \emph{likes \& dislikes} for muito baixa.
O \FC executa como um \emph{daemon} no pares da rede e permite
interações pela linha de comando ou por um \emph{app} para Android.
\end{abstract}
     
%\begin{resumo}
%\end{resumo}

\section{Introdução}

Apesar do crescimento contínuo da Internet ao longo dos anos, o seu conteúdo
disponível está cada vez mais sob o controle de poucas
empresas~\cite{internet.fixing}.
Consideramos como conteúdo qualquer tipo de informação ou interação na
Internet, tais como trocas de e-mails, consumo de notícias, interações em redes
sociais, ou até mesmo backup de documentos.
%
As empresas provedoras de conteúdo somente auxiliam o acesso entre seus
usuários, mas tipicamente não criam conteúdo original.
Por um lado, elas oferecem funcionalidades essenciais, tais como interfaces
amigáveis, armazenamento grátis e conectividade permanente.
Por outro lado, essas empresas concentram mais poder do que o necessário para
operarem, uma vez que mantém nossos dados sob controle, coletam informações
privadas, decidem o que consumimos com base em algoritmos arbitrários, exigem
conectividade mesmo para informações já consumidas, e ainda dificultam a
portabilidade entre serviços através de formatos e protocolos proprietários.

Sistemas de disseminação \emph{peer-to-peer}~\cite{p2p.survey} oferecem uma
alternativa aos serviços centralizados, eliminando intermediários e movendo
para os usuários finais toda a responsabilidade pela transferência,
armazenamento, disponibilidade e validação do conteúdo disseminado.
No entanto, há diversos novos desafios quando as responsabilidades pelo sistema
estão espalhados nas bordas da rede:
    como distribuir e replicar a infraestrutura de conectividade e
    armazenamento;
    como conectar usuários com boa performance;
    como lidar com pares (\emph{peers}) maliciosos na rede;
    e como investir na usabilidade do sistema sem um modelo de negócio.

Neste trabalho, apresentamos o \emph{\FC} como uma ferramenta para a
disseminação de conteúdo peer-to-peer sem a intermediação ou controle de acesso
centralizado, e sem a necessidade de confiança entre seus usuários.
No \FC, um usuário posta uma mensagem em um tópico, também chamado de
cadeia, e todos os outros assinantes daquela cadeia eventualmente recebem a
mensagem.
O \FC propõe duas contribuições principais: (1) um protocolo mínimo para
múltiplos arranjos de disseminação de conteúdo e (2) um sistema de reputação
autônomo e descentralizado.
%
Sobre a primeira contribuição, o \FC possibilita diversos arranjos de
disseminação de conteúdo entre seus usuários:
%
\begin{itemize}
\item \textbf{Arranjos públicos \Xon e \Xno:}
    Uma identidade pública (ex., pessoa ou organização) dissemina conteúdo para
    um público alvo (\Xon) com feedback opcional (\Xno).
    Exemplos são sites de notícias, serviços de streaming e perfis públicos em
    redes sociais.
\item \textbf{Arranjos privados \Xoo, \Xnn e \Xo:}
    Grupos de confiança (ex., amigos ou familiares) trocam mensagens privadas
    entre si.
    A comunicação pode ser em pares (\Xoo), grupos (\Xnn) ou até mesmo
    individuais (\Xo).
    Exemplos são e-mails, grupos de família e backup de documentos.
\item \textbf{Arranjos públicos \Xnn:}
    Grupos heterogêneos se comunicam publicamente. Exemplos são fóruns de
    perguntas e respostas, chats, e comércio eletrônico público.
\end{itemize}
%
%A API do protocolo é composta basicamente de comandos para assinar cadeias de
%interesse (\texttt{join}), postar novas mensagens (\texttt{post}), iterar sobre
%mensagens existentes (\texttt{traverse}), e ler uma mensagem específica
%(\texttt{get}).
%
Sobre a segunda contribuição, o sistema de reputação do \FC rege a qualidade
das postagens e autores dentro de cada cadeia e tem os seguintes objetivos:
%
\begin{itemize}
\item Combater o excesso de conteúdo restringindo o número de postagens por autor.
\item Destacar o conteúdo de qualidade com um mecanismo de \emph{likes \& dislikes}.
\item Combater SPAM, notícias falsas e conteúdo ilícito demandando reputação
      prévia de autores e removendo postagens com proporção muito baixa entre
      likes e dislikes.
\end{itemize}
%
O código fonte, documentação e vídeos introdutórios sobre o \FC estão
disponíveis em \url{https://github.com/Freechains/README}.

%A Seção~\ref{sec.related} compara o \FC com outros sistemas de
%disseminação de conteúdo.
%A Seção~\ref{sec.basic} apresenta o funcionamento básico do \FC como
%ferramenta.
%A Seção~\ref{sec.conclusion} conclui o trabalho.

\section{Comparação com Outros Sistemas}
\label{sec.related}

Diversos sistemas oferecem disseminação de conteúdo de forma distribuída.
Aqui nós consideramos \emph{middlewares publish-subscribe},
\emph{protocolos federados} e \emph{sistemas peer-to-peer completos}.

%\subsection{Middlewares Publish-Subscribe}

Os middlewares publish-subscribe (aka \emph{message brokers}) auxiliam
no desenvolvimento de aplicações descentralizadas ao desacoplar produtores
(\emph{publishers}) e consumidores (\emph{subscribers}).
Em vez de enviar as mensagens diretamente aos consumidores, os produtores as
encaminham a um intermediário (broker) que mantém uma lista de assinantes
(subscribers) que devem receber as mensagens~\cite{TODO}.
Exemplos de sistemas pubsub incluem o \emph{XMPP}, \emph{AMQP}, \emph{WebSub} e
\emph{ActivityPub}.
Um aspecto chave em \emph{pubsubs} é que, embora produtor e consumidor se
comuniquem sem conhecimento mútuo, os brokers ainda possuem um papel
centralizador na rede.
Eles são necessários para autenticar e validar as postagens, por exemplo, além
de servirem as filas de mensagens que conectam produtores e consumidores.
As filas são hospedadas em endereços previamente conhecidos (ex.,
\url{amqp://user@domain/path}), o que impede uma arquitetura inteiramente
distribuída.

\begin{comment}
Aumentar o número de servidores, como em pubsubs federados~\cite{TODO} não
afeta o modelo fundamentalmente, uma vez que cada fila se mantém independente e
serve as ações dos seus clientes isoladamente.
A peça ausente na arquitetura pubsub para torná-la totalmente descentralizada é
uma política para mesclar filas independentes, de modo que as localidades
específicas dos brokers possa variar dinamicamente sem afetar o funcionamento
do sistema.
Os problemas de sincronizar os brokers de alguma maneira são similares aos que
o \FC se propõe a resolver: como ordenar e relacionar mensagens em
múltiplos servidores, como lidar com excesso de conteúdo e SPAM, como lidar com
notícias falsas, conteúdo legal e etc.
\end{comment}

%\subsection{Protocolos e Aplicações Federadas}

Em protocolos federados~\cite{TODO}, a identidade de um usuário ainda é
atrelada a um servidor específico, mas os servidores podem se sincronizar para
permitir a comunicação externa entre seus usuários.
O serviço de e-mail (SMTP) é provavelmente o protocolo federado mais popular e
permite que o usuário de um domínio troque mensagens com um usuário de outro
domínios de maneira transparente.
Mais recentemente, o \emph{Diaspora}, \emph{Matrix} e \emph{Mastodon} atenderam
aos cenários de redes sociais, chats e \emph{microblogging} com arquiteturas
federadas.
Em sistemas federados, todo conteúdo poder ser gerenciado localmente em suas
instâncias e a sincronização externa é um passo em separado, mas que não é
necessário para o funcionamento local.
%No entanto, a identidade de um usuário não é portável, ou seja, está atrelada a
%um servidor específico e não pode ser usada transparentemente em outros.
%, que inclusive podem ter homônimos ativos.
No entanto, sem o controle de sua própria identidade, um usuário fica refém do
seu servidor.
Por exemplo, o servidor pode ser desligado pelos seus administradores ou banido
pelo resto da federação, ou ainda o próprio usuário pode ficar insatisfeito com
o serviço prestado.
Em qualquer um desses casos, o usuário terá que exportar todo o seu conteúdo e
histórico para outro servidor e anunciar a sua nova identidade aos seus
seguidores.

\begin{comment}
Em sistemas peer-to-peer como propomos com o \FC, a identidade é
controlada pelo próprio usuário com autenticação de chave pública, que pode ser
usada igualmente em qualquer par da rede.
A moderação de conteúdo é uma outra preocupação em sistemas federados.
Como exemplo, mensagens que atravessam os limites da sua federação estão
sujeitos a políticas diferentes.
Tipicamente, as regras de moderação podem ser aplicadas localmente pelo usuário
(ex., filtros de e-mail) ou ainda por servidores, o que afeta todos os seus
usuários.
O primeiro caso é mais flexível, porém é sujeito a ataques de disparo de
mensagens em massa, por exemplo.
O segundo caso pode bloquear o acesso a mensagens legítimas, pois ações globais
são potencialmente sempre mais restritivas.
No \FC, a moderação ocorre em outra dimensão.
Em vez de indivíduos ou servidores adotarem políticas verticais em todas as
suas mensagens, o \FC aplica uma política horizontal por tópico que
afeta a sua disseminação na rede peer-to-peer inteira.
Essa política parte do princípio que uma mensagem que é indesejada em uma
comunidade não necessariamente será indesejada em outras comunidades.
Como um contraponto, protocolos federados são mais apropriados para aplicações
de tempo real com um grande número de mensagens curtas, tais como chats e
streaming.
A quantidade de saltos e sobrecarga de cabeçalho pode ser bem menor em
arquiteturas cliente-servidor em comparação com sistemas peer-to-peer, que
tipicamente dependem de assinaturas digitais, \emph{hash linking} e
verificações extras.
\end{comment}

%\subsection{Sistemas Peer-to-Peer}

Em sistemas peer-to-peer, todos os pares da rede desempenham as mesmas funções
e atuam ora como clientes, ora como servidores.
O \emph{Bitcoin}~\cite{p2p.bitcoin} é provavelmente a rede peer-to-peer de
maior sucesso, mas serve a um propósito muito específico.
O \emph{Scuttlebutt}~\cite{p2p.scuttlebutt} e \emph{Aether}~\cite{TODO} atendem
aos cenários de comunicação entre amigos e grupos.
Já o IPFS~\cite{p2p.ipfs} e Dat~\cite{p2p.dat} oferecem hospedagem de arquivos
e aplicações.
O IPFS é baseado no conceito de dados imutáveis com endereçamento por conteúdo,
enquanto que o Dat em dados mutáveis com endereçamento por chave pública.
O IPFS é mais adequado para compartilhar arquivos grandes e estáveis, tais como
filmes e versões de software, enquanto que o Dat para conteúdo dinâmico como
aplicações web.
Ambos usam \emph{DHTs}~\cite{TODO} como base de suas arquiteturas, sendo
portanto otimizadas para servir conteúdos grandes e populares, mas não tanto
para buscas e descobrimento de conteúdo novo~\cite{TODO}.
%Ambos também requerem que os usuários saibam de antemão o endereço do conteúdo,
%seja o link exato de um filme, seja uma identidade em particular da rede.
Por essa razão, DHTs não seja a melhor arquitetura para modelar atualizações
contínuas de conteúdo com múltiplos autores, tais como em fóruns públicos.
%Os sistemas a seguir não usam DHTs em sua arquitetura.
\begin{comment}
O Bitcoin tem uma propriedade muito forte de garantir que as mensagens da rede
são ordenadas e que essa ordem é exatamente a mesma em todos os pares, ou seja,
que há um consenso global sobre o estado da rede.
O Bitcoin usa um algoritmo de consenso baseado em prova de trabalho que é imune
a ataques \emph{sybil} em que identidades são forjadas.
No entanto, prova de trabalho é computacionalmente cara e, na prática, mantém a
rede sob controle de poucos pares, afetando o objetivo principal de
descentralização.
O \FC usa um sistema de reputação próprio para evitar SPAM e ataques
sybil.
\end{comment}
%O sistema é apresentado na Seção XX e pode ser entendido como uma prova de
%trabalho para humanos: um autor precisa “trabalhar” para postar conteúdo de
%qualidade por sua vez é avaliado por outros humanos.
Dentre os trabalhos discutidos, o Scuttlebutt e Aether são os mais similares ao
\FC, e focam, respectivamente, na comunicação pública \Xon e \Xnn.
O Scuttlebutt é baseado em identidades de usuários que se seguem par-a-par e
formam um grafo que se reflete tanto nas conexões de rede quanto no
armazenamento redundante dos seus dados.
Como exemplo, se a identidade \emph{A} segue a identidade \emph{B}, então o
computador de \emph{A} se conecta diretamente ao computador de \emph{B} e
também armazena todas as postagens de \emph{B} localmente.
%As relações são assimétricas, permitindo que \emph{B} não siga \emph{A}
%necessariamente no exemplo dado.
O \FC se baseia em tópicos, mas como um tópico pode ser de propriedade de uma
identidade, então é possível modelar o mesmo comportamento do Scuttlebutt.
%A topologia de rede do \FC também se baseia em conexões par a par, mas
%não necessariamente entre seguidores.
Já o Aether se baseia em comunidades organizadas em torno de tópicos.
As postagens são replicadas na rede mas são efêmeras considerando uma janela de
tempo.
O Aether usa prova de trabalho nas postagens para evitar SPAM.
Além disso, as comunidades elegem moderadores para prevenir comportamento
abusivo.
O \FC introduz um sistema de reputação distribuído para lidar com essas
ameaças conforme discutido na Seção~\ref{sec.reps}.

\section{Visão Geral do \FC}
\label{sec.freechains}

\subsection{Funcionamento Básico}

O \FC é um sistema publish-subscribe baseado em tópicos: um usuário
posta uma mensagem em um tópico e seus assinantes eventualmente recebem a
mensagem.
A Figura~\ref{fig.all} ilustra os quatro conceitos básicos do \FC:
\emph{cadeias}, \emph{blocos}, \emph{autores} e \emph{pares}.

\begin{figure}[ht]
\centering
\includegraphics[width=.75\textwidth]{all.png}
\caption{Conceitos básicos do \FC: cadeias, blocos, autores e pares.}
\label{fig.all}
\end{figure}

As \emph{cadeias}, destacadas em quadrados coloridos (vermelho, verde e
amarelo), são independentes umas das outras e podem representar, por exemplo,
os \emph{tweets} de uma personalidade (arranjo \Xon público), um grupo de
\emph{Whatsapp} de amigos (arranjo \Xnn privado) ou um \emph{subreddit} em
torno de um interesse em comum (arranjo \Xnn público).

Cada cadeia possui um conjunto de \emph{blocos} com as postagens, que estão
destacadas como elipses coloridas (roxo, vermelho e verde), em que cada cor
representa um \emph{autor}.
Os blocos da cadeia formam um grafo de causalidade indicando a relação temporal
entre elas.
Por exemplo, a primeira mensagem do autor vermelho aconteceu antes das
primeiras mensagens dos autores roxo e verde.
O grafo é um DAG e expressa a ordem parcial entre todos os blocos na cadeia.
Toda cadeia tem um bloco \emph{gênesis} preexistente (não exibido na imagem)
que, por construção, é anterior a (e alcançável por) todos os blocos da cadeia.

O grafo da cadeia é replicado em todos os pares assinantes, ou seja, todos os
nós da rede mantém todas as mensagens da cadeia, independentemente da autoria
das postagens.
Na figura, a cadeia amarela é replicada em três pares.
A disseminação do grafo pela rede é feita por \emph{gossip}, ou seja, os pares
se conectam dois a dois para sincronizar os seus grafos.
Dessa maneira, o recebimento de mensagens no pares é eventual, pois depende de
um roteamento par a par da origem ao destino.

O \FC executa como um servidor ou \emph{daemon} e escuta requisições de
usuários locais, por exemplo para postar ou ler mensagens, e se comunica com
outros pares da rede peer-to-peer para sincronizar as suas cadeias de
interesse.
O daemon pode ser acessado de três formas equivalentes:
%
\begin{itemize}
\item Pela linha de comando, que usamos no resto do artigo.
\item Por uma API em Kotlin, que pode ser usada para criação de novas aplicações.
\item Por um protocolo textual, que permite usar outras linguagens via sockets.
\end{itemize}
%
\begin{comment}
% 
Os principais comandos do \FC são enumerados a seguir:
%
\begin{itemize}
\item {\footnotesize\texttt{freechains-host start}}:     inicia daemon no diretório passado
\item {\footnotesize\texttt{freechains crypto}}:         cria identidade criptográfica
\item {\footnotesize\texttt{freechains chains join}}:    inscreve-se na cadeia
\item {\footnotesize\texttt{freechains chain post}}:     posta na cadeia
\item {\footnotesize\texttt{freechains chain get}}:      lê postagem da cadeia
\item {\footnotesize\texttt{freechains chain traverse}}: itera sobre a cadeia
\item {\footnotesize\texttt{freechains peer send/recv}}: sincroniza com par da rede
\end{itemize}
%
Exceto pelo comando para iniciar o daemon, os outros comandos principais são
efetuados remotamente através do cliente \texttt{freechains}.
Isso permite compartilhar um par da rede entre múltiplos clientes, por exemplo,
um desktop e um celular.
\end{comment}
%
A sequência de comandos a seguir ilustra o funcionamento básico do \FC:
%
{\footnotesize
\begin{verbatim}
$ freechains start /var/fcs/ &      # inicia daemon em background
$ freechains crypto pubpvt "senha"  # cria identidade publica
EB172E... 96700A...                 # (retorna chaves publica e privada)
$ freechains chains join "#chat"    # se inscreve na cadeia #chat
$ freechains chain "#chat" post inline "Olá Mundo!" --sign=96700A...
\end{verbatim}
}

Após iniciar o daemon, o usuário cria uma identidade para si.
A chave pública pode ser compartilhada e a chave privada deve ser guardada em
segredo.
Em seguida o usuário se inscreve na cadeia \texttt{\#chat} e posta uma mensagem
assinando-a com a sua chave privada.
Até aqui, tudo ocorre localmente na máquina com o daemon em execução.
A comunicação com outros pares deve ser feita explicitamente:
%
{\footnotesize
\begin{verbatim}
$ freechains peer 10.0.0.2 send "#chat"  # envia  #chat para 10.0.0.2
$ freechains peer 10.0.0.2 recv "#chat"  # recebe #chat  de  10.0.0.2
\end{verbatim}
}
%
Para ler as postagens de uma cadeia, é necessário primeiro identificar os
blocos não lidos:
%
{\footnotesize
\begin{verbatim}
$ freechains chain "#chat" genesis                # código do genesis
10EEB7...
$ freechains chain "#chat" traverse all 0_10EE... # todos ate o genesis
1_A5EF... 2_1B5C... 2_2144...
$ freechains chain "#chat" get payload A563EF...  # conteudo do bloco
Bom dia!
\end{verbatim}
}

O comando \texttt{traverse} retorna todos os blocos mais novos que o(s)
bloco(s) passados como parâmetro e serve para identificar as postagens que
ainda não foram lidas.
Inicialmente, usamos a identificação do bloco gênesis como âncora, já que
nenhum bloco ainda foi lido.
Em seguida, é possível usar o comando \texttt{heads} para identificar os blocos
mais novos na "cabeça" do grafo e usá-los como âncora na próxima sincronização.
%
Uma outra forma é checar o conteúdo em tempo real através do comando
\texttt{listen}:
%
{\footnotesize
\begin{verbatim}
$ freechains chain "#chat" listen   # escuta novos blocos em tempo real
1_A563...                           # 1o bloco recebido
2_12AB...                           # 2o bloco recebido
2_FA21...                           # 3o bloco recebido
\end{verbatim}
}
%
O comando é bloqueante e exibe uma nova linha, sempre que um novo bloco chega,
com a sua identificação.
Esse comando pode ser usado em conjunto com um \emph{pipe} para reagir a novas
mensagens em tempo real.
%
Note que os comandos de \texttt{send} e \texttt{recv} devem ser executados para
cada cadeia e para cada par da rede de interesse.
Por exemplo, se o usuário segue 10 cadeias e se liga a 5 pares, os comandos
deverão ser executados 100 vezes a cada sincronização ($2\times10\times5$).
O \FC inclui uma ferramenta para automatizar o processo inteiro de
sincronização.
É possível registrar pares e cadeias e a ferramenta usa os comandos
\texttt{listen}, \texttt{send} e \texttt{recv} para sincronizar-se assim que
uma nova mensagem é postada ou recebida.

Mais tecnicamente, um bloco é uma estrutura de dados que persiste uma única
mensagem na cadeia.
Ele se encadeia às postagens anteriores e é encadeado pelas próximas postagens.
A Figura~\ref{fig.block} ilustra o encadeamento e o trecho a seguir mostra a
representação interna de um bloco:
%
{\footnotesize
\begin{verbatim}
data Block:
    msg   : String              ; (1)
    sign? : (HKey,String)       ; (2)
    meta  :                     ; (3)
        time  : Long
        msg   : (Bool,Hash)
        like? : (Hash,Int)
        backs : Array<Hash>
    hash  : Hash                ; (4)
\end{verbatim}
}
%
As quatro partes enumeradas acima são as seguintes:
%
\begin{enumerate}
\item[msg:]  O conteúdo da mensagem propriamente dita.
\item[sign:] Uma assinatura opcional com a chave pública do autor e o código da
             assinatura.
\item[meta:] Os metadados da mensagem: com tempo de criação, se está
             criptografada, com o seu hash, se é um like/dislike, e os elos
             para blocos anteriores.
\item[hash:] O hash criptográfico dos metadados do bloco que o identificam
             univocamente. O hash é prefixado com a altura do bloco na cadeia.
             Por exemplo, o bloco
                {\scriptsize\texttt{5\_F700CC98A6BA6A562CF6272AFC1044CB0F049E2E71D1076DA3391E85EE2CE2B8}}
             possui altura \texttt{5} e hash \texttt{F700CC...}.
\end{enumerate}

\begin{figure}[ht]
\centering
\includegraphics[width=.75\textwidth]{block.png}
\caption{Um bloco (centro) se encadeia aos anteriores (esquerda) e é encadeado
         pelos próximos (direita).}
\label{fig.block}
\end{figure}

O encadeamento dos blocos forma um grafo direcionado acíclico que é imune a
modificações (\emph{Merkle DAG}).
Ao se criar um novo bloco, calcula-se o hash dos seus metadados, que por sua
vez incluem os hashes da mensagem em si e dos blocos anteriores.
Esse novo hash identifica o bloco em todas as operações, inclusive nos elos dos
próximos blocos, e pode a qualquer momento ser recalculado para verificar se os
dados sofreram modificações.
Como os elos anteriores também são identificados da mesma forma, o grafo pode
ser verificado até o seu gênesis, cujo hash depende somente dos parâmetros da
cadeia, sendo igualmente verificável.

\subsection{Arranjos de Disseminação de Conteúdo}

O \FC suporta três tipos de cadeias, cada uma com um propósito diferente:
%
\begin{itemize}
\item \textbf{Fóruns Públicos}:
    Arranjo \Xnn público.
    Comunicação pública entre participantes sem confiança mútua.
    Exemplos: fóruns de perguntas e respostas, chats, e comércio eletrônico
              público.
\item \textbf{Grupos Privados}:
    Arranjos privados \Xoo, \Xnn e \Xo.
    Comunicação privada entre pares, grupos ou individuais.
    Exemplos: e-mail, grupos de WhatsApp, backup.
\item \textbf{Identidade Pública}:
    Arranjos públicos \Xon e \Xno.
    Uma identidade pública (ex., pessoa ou organização) dissemina conteúdo para
    um público alvo (\Xon) com feedback opcional (\Xno).
    Exemplos: sites de notícias, serviços de streaming e perfis públicos em
    redes sociais.
\end{itemize}
%
O tipo da cadeia é determinado pelo prefixo em seu nome:
%
\begin{itemize}
    \item \textbf{\texttt{\#}}: fórum público (ex., \texttt{\#chat})
    \item \textbf{\texttt{\$}}: grupo privado (ex., \texttt{\$familia})
    \item \textbf{\texttt{@}}: identidade pública (ex., \texttt{@B2853F...})
\end{itemize}

Em fóruns públicos, as mensagens circulam entre usuários e pares sem confiança
mútua.
Por essa razão, cadeias desse tipo dependem do sistema de reputação do \FC para
serem viáveis sob completa descentralização.
%
Em grupos privados, todas as postagens são automaticamente criptografadas
usando uma chave compartilhada entre pares de confiança.
Nesse caso, o comando \texttt{join} recebe um parâmetro extra com a chave, que
deve ser executado da mesma forma em todos os pares:
%
{\footnotesize
\begin{verbatim}
$ freechains chain "$familia" 8889BB...
\end{verbatim}
}
%
Todos os usuários de grupos privados têm reputação infinita e nem é necessário
assinar as mensagens.
%
Para cadeias de identidade pública, o nome deve ter o prefixo \texttt{@}
seguido pela chave pública do autor proprietário da cadeia.
O proprietário tem reputação infinita e deve assinar todas as mensagens com a
sua chave privada.

\subsection{Sistema de Reputação}

Fóruns públicos descentralizados são um convite para usuários maliciosos
abusarem da rede com SPAM, notícias falsas e conteúdo ilícito.
O sistema de reputação do \FC permite somente que usuários com reputação prévia
possam postar conteúdo novo em uma cadeia.
Caso contrário, a postagem fica retida no par de origem e precisa ser aprovada
por algum usuário com reputação para ser disseminada na rede.
Cada cadeia é controlada por um sistema autônomo que contabiliza a quantidade
de likes e dislikes a autores e postagens.
Como cada cadeia é independente, a reputação de um autor pode variar entre
elas.
A unidade de reputação é conhecida como $rep$ e pode ser gerada, consumida e
transferida de diversas formas:
%
\begin{enumerate}
\item Geração:
    \begin{enumerate}
    \item A primeira postagem de uma cadeia adiciona $+30~reps$ ao autor.
    \item Qualquer postagem mais antiga que $24h$ conta $+1~rep$ ao autor, mas
          limitada a uma por dia. Se o autor postou 10 postagens nos últimos 7
          dias, ele recebe somente $+7~reps$.
    %, aka \emph{postagem consolidada},
    \end{enumerate}
\item Consumo:
    \begin{enumerate}
    \item Qualquer postagem mais jovem que $24h$ conta $-1~rep$ ao autor.
    %, aka \emph{postagem nova},
    \end{enumerate}
\item Transferência:
    \begin{enumerate}
    \item Um \emph{like}    partindo do autor \emph{A} à postagem \emph{P} do
          autor \emph{B} conta $-1~rep$ para \emph{A} e $+1~rep$ para \emph{B}.
    \item Um \emph{dislike} partindo do autor \emph{A} à postagem \emph{P} do
          autor \emph{B} conta $-1~rep$ para \emph{A} e $+1~rep$ para \emph{B}.
          Se uma postagem alcança pelo menos $5$ dislikes e o dobro do número
          de likes, então o seu conteúdo é escondido da rede.
    \end{enumerate}
\item Regras Adicionais:
    \begin{enumerate}
    \item Postagens de usuários sem reputação ficam retidas (não são nem
          encadeadas nem retransmitidas) até receberem um like.
    \item Um usuário fica limitado a $+30~reps$.
    \item Para todos os efeitos, somente postagens mais novas que 90 dias são
          consideradas.
    \item Em cadeias privadas, todos os usuários têm reputação infinita.
    \item Em cadeias de identidade pública, o proprietário tem reputação
          infinita.
    \end{enumerate}
\end{enumerate}
%
A primeira regra de geração de $reps$ (\texttt{1.a}) é essencial para fazer o
``bootstrap'' de uma cadeia, uma vez que seria impossível realizar postagens se
ninguém possui reputação nenhuma.
Assim, o autor da primeira postagem molda a cultura inicial da cadeia ao
transferir sua reputação a outros autores, que por sua vez transferem a outros
autores, expandindo a comunidade em alguma direção de preferência.
%
Note que cadeias de mesmo nome mas com primeiros autores diferentes são
incompatíveis e o protocolo se recusa a sincronizá-las.
Isso pode acontecer quando duas redes independentes (ex., UERJ e PUC-Rio)
seguem uma cadeia de nome usual (ex., \texttt{\#computacao}) e, de algum jeito,
acabam se unindo através de um par em comum.

Os comandos de \texttt{like} e \texttt{dislike} atuam sobre uma postagem já
existente na cadeia:
%
{\footnotesize
\begin{verbatim}
$ freechains chain "#chat" like 12AB5C... --sign=96700A...
\end{verbatim}
}
%
Nesse caso, o usuário que assinou o like transfere $1~rep$ seu para o autor da
postagem referenciada (regra~\texttt{3.a}).
%
Já o comando \texttt{reps} verifica a reputação passando a chave pública do
autor ou identificador hash da postagem a ser consultada:
%
{\footnotesize
\begin{verbatim}
$ freechains chain "#chat" reps 12AB5C...
5   <-- reputacao da postagem
\end{verbatim}
}

A qualidade das postagens é subjetiva e cabe aos usuários as julgarem com
likes, dislikes ou simples abstenções.
Um usuário pode desgostar de uma postagem por considerá-la ofensiva, SPAM,
falsa, ilícita, ou por simples desacordo.
Por um lado, como os $reps$ são finitos, os usuários devem ponderar e evitar o
seu gasto indiscriminado.
Por outro lado, os $reps$ também expiram após 3 meses, então os usuários tem
incentivos para cooperar com a qualidade das cadeias.
Um conteúdo pode ser banido quando o número de dislikes supera em muito o
número de likes.
Considerando que os $reps$ são escassos, o banimento de postagens não tem o
objetivo de eliminar discordâncias de opinião, mas sim de evitar a atuação de
usuários maliciosos.

O sistema de reputação do \FC busca oferecer oportunidades de participação nas
cadeias minimamente justas.
Por isso, restringe o número de postagens de um dado autor de duas maneiras:
    (1) penaliza postagens com menos de $24h$ (regra~\texttt{2.a}); e
    (2) limita o ganho de reputação por postagens novas em $1 rep$ por dia
        (regra~\texttt{1.b}).
A primeira maneira previne que um mesmo autor poste muitas mensagens em
sequência sob a pena de consumir a sua própria reputação muito rapidamente.
A segunda evita o acúmulo de reputação simplesmente por postar com muita
frequência.

O tamanho da ``economia'' de uma cadeia é a sua quantidade de postagens
consolidadas (regra~\texttt{1.b}), dado que postagens com mais de $24h$ são a
única forma de gerar $reps$.
Isso porque likes e dislikes apenas transferem reputação e a reputação inicial
se torna insignificante com o tempo.
A economia também depende muito da quantidade de autores ativos, uma vez que
a geração de $reps$ por autor é limitada a $1$ por dia.
Esse mecanismo incentiva o acolhimento de novos autores a contribuírem, ao
mesmo tempo que desincentiva dislikes pois drenam $2~reps$ da economia
(regra~\texttt{3.b}).
Por um lado, esse desincentivo contribui para discussões com um nível razoável
de desacordo, pois evita o colapso da cadeia com um surto de dislikes.
Por outro lado, conteúdos claramente indesejados como SPAM de usuários que
pouco contribuíram são banidos rapidamente da cadeia com poucos dislikes
(regra~\texttt{3.b}).

\section{Conclusão}
\label{sec.conclusion}

TODO

\begin{comment}
O \FC é um protocolo peer-to-peer para disseminação de conteúdo que pode ser
manipulado através de uma ferramenta para criar aplicações distribuídas.
Aplicações
    - wiki
    - face

\section{Figures and Captions}\label{sec:figs}

Figure and table captions should be centered if less than one line
(Figure~\ref{fig:exampleFig1}), otherwise justified and indented by 0.8cm on
both margins, as shown in Figure~\ref{fig:exampleFig2}. The caption font must
be Helvetica, 10 point, boldface, with 6 points of space before and after each
caption.

\begin{figure}[ht]
\centering
\includegraphics[width=.3\textwidth]{fig2.jpg}
\caption{This figure is an example of a figure caption taking more than one
  line and justified considering margins mentioned in Section~\ref{sec:figs}.}
\label{fig:exampleFig2}
\end{figure}

In tables, try to avoid the use of colored or shaded backgrounds, and avoid
thick, doubled, or unnecessary framing lines. When reporting empirical data,
do not use more decimal digits than warranted by their precision and
reproducibility. Table caption must be placed before the table (see Table 1)
and the font used must also be Helvetica, 10 point, boldface, with 6 points of
space before and after each caption.

\begin{table}[ht]
\centering
\caption{Variables to be considered on the evaluation of interaction
  techniques}
\label{tab:exTable1}
\includegraphics[width=.7\textwidth]{table.jpg}
\end{table}

\section{Images}

All images and illustrations should be in black-and-white, or gray tones,
excepting for the papers that will be electronically available (on CD-ROMs,
internet, etc.). The image resolution on paper should be about 600 dpi for
black-and-white images, and 150-300 dpi for grayscale images.  Do not include
images with excessive resolution, as they may take hours to print, without any
visible difference in the result. 

\section{References}

Bibliographic references must be unambiguous and uniform.  We recommend giving
the author names references in brackets, e.g. \cite{knuth:84},
\cite{boulic:91}, and \cite{smith:99}.

The references must be listed using 12 point font size, with 6 points of space
before each reference. The first line of each reference should not be
indented, while the subsequent should be indented by 0.5 cm.

\end{comment}

\bibliographystyle{sbc}
\bibliography{sbc-template}

\end{document}
